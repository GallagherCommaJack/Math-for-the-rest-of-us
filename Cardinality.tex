\documentclass[xcolor=svgnames]{beamer}

\usepackage[utf8]    {inputenc}
\usepackage[T1]      {fontenc}
\usepackage[english] {babel}

\usepackage{amsmath,amsfonts,graphicx}
\usepackage{beamerleanprogress}


\title
  [Counting infinity\hspace{2em}]
  {Counting infinity}

% \subtitle
%   {Cardinals, ordinals, and why $\infty + 1 \neq 1 + \infty$ }

\author
  [Jack Gallagher]
  {Jack Gallagher}

\date{May 11, 2015}

\institute{Mentoring Academy}

\begin{document}

\maketitle

\section{Subsets and size}

\begin{frame}{}
  Is a subset smaller than the whole? \pause

  Why should it be? \pause

  Any counterexamples?
\end{frame}

\begin{frame}{}
  Are the naturals bigger than the evens? \pause

  Arguments for:
  \begin{itemize}
    \item It's a subset \pause
    \item The ``well duh'' argument \pause
  \end{itemize}

  Can we formalize this? \pause No, because it's \emph{false}!
\end{frame}

\section{Pidgeons and hotels}

\begin{frame}{}
  \begin{center}
    \Huge An experiment in hotel management
  \end{center}
\end{frame}

\section{Countability and Numbers}

\begin{frame}{}
  \begin{center}
    \Huge Are any sets \emph{bigger} than the naturals?
  \end{center}
\end{frame}

\begin{frame}{}
  \begin{center}
    \Huge Integers?
  \end{center}
\end{frame}

\begin{frame}{}
  \begin{center}
    \Huge Rational numbers?
  \end{center}
\end{frame}

\begin{frame}{}
  \begin{center}
    \Huge The reals?
  \end{center}
\end{frame}

\section{Counting the reals}

\begin{frame}{}
  \begin{center}
    \Huge Diagonalization
  \end{center}
\end{frame}

\begin{frame}{}
  One way to define the reals is arbitrary digit sequences \pause
  \begin{align}
    \frac{1}{3} & = 0.33333 \dots \\
       \pi      & = 3.14159 \dots \\
       e        & = 2.71828 \dots \\
       \sqrt{2} & = 1.41421 \dots \\
                & \cdots
  \end{align} \pause

  Can we count these sequences?
\end{frame}

\end{document}
