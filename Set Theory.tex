\documentclass[xcolor=svgnames]{beamer}

\usepackage[utf8]    {inputenc}
\usepackage[T1]      {fontenc}
\usepackage[english] {babel}

\usepackage{amsmath,amsfonts,graphicx}
\usepackage{beamerleanprogress}


\title
  [Set Theory\hspace{2em}]
  {Set Theory}

\subtitle
  {The Implicit Foundation of Mathematics}

\author
  [Jack Gallagher]
  {Jack Gallagher}   

\date{April 2, 2015}

\institute{Mentoring Academy}

\begin{document}

\maketitle

\section{Introduction}

\begin{frame}{}
  \begin{center}
  \Huge Why do we need foundations?
  \end{center}
\end{frame}

\begin{frame}{}
    I'd contend there are three real reasons:
    \begin{enumerate}
    \item Unambiguity
    \item Checkability
    \item Automation?
    \end{enumerate}
\end{frame}

\section{Sets}

\begin{frame}{Sets}

    Sets are \emph{collections of objects}

    \begin{itemize}
    \item $\{1,2,3\}$
    \item $\{1,2,\frac{1}{2}\}$
    \item $(1,10)$, $[1,10]$
    \end{itemize} \pause

    Sets can also contain other sets
    
    \begin{itemize}
    \item $\{1,2,\{1\}\}$
    \item $\{1,\{\}\}$
    \end{itemize}

\end{frame}

\begin{frame}{Basic Properties of Sets}

    Unordered
        \begin{itemize}
        \item $\{1,2\} = \{2,1\}$
        \end{itemize} \pause
    Unique
        \begin{itemize}
        \item $\{1,1\}$ isn't a set, it's just $\{1\}$
        \end{itemize} \pause
    Testable
        \begin{itemize}
        \item $1 \in \{1,2,3\}$
        \item $1 \notin \{2,3,4\}$
        \end{itemize} \pause


\end{frame}

\begin{frame}{Set Builder Notation}
    
    Another way of specifying sets
    \begin{itemize}
    \item $\{x\ |\ x < 3\} = (-\infty,3)$ 
    \item $\{x\ |\ x^2 = 1\} = \{-1,1\}$
    \item $\{x\ |\ x \in (-1,1),\ x + 1 = 0\} = \{\}$
    \end{itemize}

\end{frame}

\section{Oops!}

\begin{frame}{}
    \begin{center}
    \Huge Paradox alert!
    \end{center}
\end{frame}

\begin{frame}{Russel's Paradox}
    $$\text{Let } A = \{x\ |\ x \notin x\}$$
    The question is, is $A$ contained in $A$?
    $$A \in A \implies A \notin A$$ \pause
    $$A \notin A \implies A \in A$$ \pause

    Either way, we have a paradox!
    \pause
    
    Beginning of the ``crisis of foundations''
\end{frame}

\begin{frame}{How to fix this?}
    Several patching methods
    
    Most successful has been to weaken comprehension
    \pause
    
    $\{x\ |\ x \notin x\}$ is no longer a set,
    but $\{x\ |\ x \in A, x \notin x\}$ for some set $A$ is. \pause

\end{frame}

\begin{frame}{Lessons from the paradox}
    \begin{itemize}
    \item Our naive idea of a set was too broad, needed more formality
    \item Big motivator for foundations
    \item What's the smallest possible set of axioms that captures all of math?

    \end{itemize}
\end{frame}

\section{Number theory}

\begin{frame}{}
    \begin{center}
    \Huge Encoding number theory
    \end{center}
\end{frame}

\begin{frame}{Some operations on sets}
    Set union \pause
    \begin{itemize}
    \item $\{1,2\} \cup \{2,3\} = \{1,2,3\}$
    \item $\{\} \cup X = X \cup \{\} = X$ \pause
    \item $\{x\ |\ P(x)\} \cup \{x\ |\ Q(x)\} = \{x\ |\ P(x) \text{ or } Q(x)\}$
    \end{itemize} \pause
    
    Set intersection \pause
    \begin{itemize}
    \item $\{1,2\} \cap \{2,3\} = \{2\}$
    \item $\{\} \cap X = X \cap \{\} = \{\}$ \pause
    \item $\{x\ |\ P(x)\} \cap \{x\ |\ Q(x)\} = \{x\ |\ P(x) \text{ and } Q(x)\}$
    \end{itemize}
\end{frame}

\begin{frame}{Encoding number theory}
    Natural numbers
    \begin{itemize}
    \item $0 = \{\}$
    \item $\sigma(n) = n \cup \{n\}$
    \item $1 = \{0\} = \{\{\}\}$
    \item $2 = \{0,1\} = \{\{\}, \{\{\}\}\}$ \dots
    \end{itemize}

    Addition
    \begin{itemize}
    \item $A + B = \sigma^{|A|}(B)$
    \end{itemize}
    
    Multiplication then gets the familiar definition, as does exponentiation
\end{frame}

\begin{frame}{Encoding number theory?}
    \begin{center}
    \Huge Does this definition capture all of number theory?
    \end{center}
\end{frame}

\end{document}